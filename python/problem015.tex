\documentclass{article}
\usepackage{amsmath}

\begin{document}

\title{Problem 15}
\maketitle

This problem was not particularly difficult, but it still took me some time to solve because I'm not very good with combinatorics. I recalled some concepts from high school, but none of them immediately helped me see the solution.

The main challenge for me was realizing that the order of the steps did not matter. Initially, I attempted to calculate $ 40! $, thinking that there were 20 moves in the "x direction" and 20 moves in the "y direction." However, it didn't work because each step in a given direction is a step in a given direction, that is, they're the same. In other words, swapping two identical moves does not create a new unique path.

At this point, I recalled the combination formula:

$$
\binom{n}{r} = \frac{n!}{r!(n-r)!}
$$

This formula counts the number of ways to choose \( r \) elements from a set of \( n \) elements, without considering order. However, I needed a version of this formula that would account for a set consisting of two distinct groups of elements. This led me to the more general form:

$$
\binom{n+m}{n} = \frac{(n+m)!}{n! \, m!}
$$

This equation represents the number of ways to arrange \( n \) steps in one direction and \( m \) steps in another, considering that the order of moves within each direction does not matter.

I'll admit that I had to look up this specific formula because I wouldn’t have recalled it on my own. Nevertheless, I really enjoyed working through this problem and gaining a deeper understanding of how combinatorial principles apply to it!

\end{document}
